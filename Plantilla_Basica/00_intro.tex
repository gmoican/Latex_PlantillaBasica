\section{Introducción}

Esta plantilla de LaTeX se ha creado con el propósito de facilitar la elaboración de informes de prácticas de forma más sencilla y elegante en comparación a editores de texto convencionales. La base de esta plantilla nace del trabajo de \href{https://github.com/jmrplens/TFG-TFM_EPS}{José María Plens} orientado a la elaboración de TFG para la Escuela Politécnica Superior de la Universidad de Alicante; las diferencias entre estos trabajos son que el original contiene muchos más elementos mientras que en esta versión se han eliminado librerías, comandos y recursos específicos para dejar una plantilla básica.

Si necesitas una ayuda específica sobre cómo trabajar con LaTeX, te recomiendo que consultes el enlace superior así como las ayudas de Overleaf, pero si quieres empezar a redactar ya, sigue estas instrucciones y mira por encima el resto del documento:

\begin{itemize}
    \item En Overleaf, carga todos estos ficheros.
    \item Ve a Menu y asegúrate que el compilador es XeLaTeX y el archivo principal es \textit{main.tex}.
    \item En el archivo \textit{main.tex} solo tienes que poner el título del trabajo, autor y fecha. Después de eso se cargarán los archivos correspondientes a cada sección con el comando \verb|\include{}|; si quieres que cada sección comience en la misma página que donde acabó la sección anterior, cambia este comando por \verb|\input{}|.
    \item Modifica cada sección desde su fichero correspondiente, es más práctico y eficiente que tener todo tu documento en un mismo fichero. Si además creas \verb|\section{}| y \verb|\subsection{}|, se te irá generando un índice en la esquina inferior-izquierda de Overleaf y podrás ir rápidamente a cada apartado.
\end{itemize}